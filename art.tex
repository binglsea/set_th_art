\documentclass[zihao=-4,a4paper]{ctexart}
\usepackage{geometry}
\geometry{a4paper,total={171.8mm,246.2mm},left=19.1mm,top=25.4mm,}

\title{凯莱《一般拓扑学》公理集合论附录\\ 的翻译纠误与商榷}
\author{龙\quad 冰}
\date{}

\begin{document}
\maketitle
\begin{abstract}
科学出版社数学名著译丛 对 John L. Kelley 《 一般拓扑学》一书的 1982 年中文版, 全新排版出版了 2010 年第二版.
本文讨论新版中对序数和分类公理模式等仍未纠正的翻译错误.
\end{abstract}
	
关键词: 集合 (set); 类 (class); 序类 (ordinal); 
	序数 (ordinal number); 
	分类公理模式 (classification axiom-scheme).
	
	
	
\section{导言}
科学出版社1982 年在数学名著译丛中首度出版了吴从炘和吴让泉翻译的凯莱  (John L. Kelley) 《 General  Topology》\cite{jkelley1975}一书的中文版 《 一般拓扑学》.时隔 28 年, 2010 年又全新排版出版了中文 第二版\cite{jkelley2010zh}, 可见原著对一般拓扑学领域的发展影响甚大. 反映在教材编写上也是如此\cite{PuJiangHu1985}.译者两版中文译本的工作无疑是值得肯定的. 不太为人知的是, Kelley 该书关于公理集合论附录的重要意义. 该书第 0 章预备知识介绍了一般拓扑学所需的 (非公理化的) 集合论知识, 大部分读者满足于此. 
附录将第 0 章的集合论浓缩式地公理化, 虽然称之为初等集合论, 却艰深难懂. 领略一下, 公理集合论附录的英文原文 31 页. 讲了集合的各种运算和大部分公理, 花了整整 21 页后读者才在无穷公理中见识到了真正的集合存在. 之前只是说, 属于某个类的类是集合, 但没有见到集合, 只是假定, 如果集合存在的话怎样怎样. 如此, 必然有大部分读者敬而畏之, 望而止步. 

事实上, 凯莱《 一般拓扑学》 的公理集合论附录独树一帜, 影响不可忽视. 对 ZFC 公理集合论\cite{jjiang1991}\cite{enderton1977}的扩展主要有三大学派\cite{wikiSetTheory}, 它们都在类下面讨论集合, 而 ZFC 则不直接涉及类. Von Neumann–Bernays–Gödel 集合论\cite{wikiNBGSetTheory}是 ZFC 的保守扩展,  它使用有限公理模式, 得到和 ZFC 相同的结果. 
Morse-Kelley 集合论\cite{wikiMKSetTheory} 和 Tarski–Grothendieck 集合论\cite{wikiTGSetTheory} 都比 ZFC 更强, 后者加了 Tarski 公理以支持范畴论. 凯莱一书的附录概括总结了 Morse-Kelley 集合论,
这方面的研究开始于王浩的工作\cite{wang1949}. Morse-Kelley 集合论与 Von Neumann–Bernays–Gödel  集合论不同, 不能有限公理化.但在它的框架下可以证明 ZFC 和 Neumann–Bernays–Gödel 是一致的.

鉴于凯莱一书公理集合论附录的不应忽视的重要性和本身的艰深难读, 读者自然期望其中文翻译能妙笔生花, 使名著的阅读更为容易. 
	
\section{序类和序数 }
	
	

	
\begin{thebibliography}{9}
	\bibitem{jkelley1975}
John L. Kelley,  \emph{General Topology}, Graduate Texts in Mathematics 27, ISBN 9780387901251, 1975, Springer Verlag.
	
	\bibitem{jkelley2010zh}
J. L. 凯莱,  \emph{一般拓扑学}, 数学名著译丛, 吴从炘 / 吴让泉译, ISBN 9787030271181, 2010, 科学出版社.

	\bibitem{PuJiangHu1985}
蒲保明, 蒋继光, 胡淑礼, \emph{拓扑学}, 1985, 高等教育出版社.

	\bibitem{jjiang1991}
蒋继光, \emph{一般拓扑学专题选讲}, ISBN 7540812842, 1991, 四川教育出版社.
	
	\bibitem{enderton1977}
H. B. Enderton, \emph{Elements of Set Theory}, ISBN 9780122384400, 1977, Academic Press.
	
	\bibitem{wang1949}
Hao Wang, \emph{On Zermelo's and von Neumann's axioms for set theory}, Proc. Natl. Acad. Sci. U.S.A., 35 (3): 150–155.

	\bibitem{wikiSetTheory}
基维百科, 集合论,	https://en.wikipedia.org/wiki/Set\_theory

	\bibitem{wikiMKSetTheory}
基维百科, Morse–Kelley 集合论,	https://en.wikipedia.org/wiki/Morse–Kelley\_set\_theory

   \bibitem{wikiNBGSetTheory}
基维百科, Von Neumann–Bernays–Gödel  集合论,\\ https://en.wikipedia.org/wiki/Von\_Neumann–Bernays–Gödel\_set\_theory

   \bibitem{wikiTGSetTheory}
基维百科, Tarski–Grothendieck  集合论,\\
https://en.wikipedia.org/wiki/Tarski–Grothendieck\_set\_theory

\end{thebibliography}
\end{document}
