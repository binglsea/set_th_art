\documentclass[zihao=-4,a4paper]{ctexart}
\usepackage{geometry}
\geometry{a4paper,total={171.8mm,246.2mm},left=19.1mm,top=25.4mm,}

\title{凯莱《一般拓扑学》公理集合论部分翻译的纠误}
\author{龙\quad 冰}
\date{}

\begin{document}
\maketitle
\begin{abstract}
科学出版社数学名著译丛 对 John L. Kelley 《 一般拓扑学》一书的 1982 年中文版, 全新排版出版了 2010 年第二版.
本文讨论新版中对序数和分类公理方案等仍未纠正的翻译错误.
\end{abstract}
	
关键词: 集合 (set); 类 (class); 序类 (ordinal); 
	序数 (ordinal number); 
	分类公理方案 (classification axiom-scheme).
	
	
	
\section{导言}
科学出版社1982 年在数学名著译丛中首度出版了吴从炘和吴让泉翻译的  John L. Kelley 的 《 General  Topology》\cite{jkelley1975}一书的中文版 《 一般拓扑学》, 2010 年又全新排版出版了中文 第二版\cite{jkelley2010zh}. Kelley 的原著对一般拓扑学领域的发展影响甚大, 反映在教材编写上也是如此\cite{PuJiangHu1985}.译者两版中文译本的工作无疑是值得肯定的. 不太为人知的是, Kelley 该书关于公理集合论的附录也是独树一帜, 影响不小. 

	
\section{序类和序数 }
	
	

	
\begin{thebibliography}{9}
	\bibitem{jkelley1975}
John L. Kelley,  \emph{General Topology}, Graduate Texts in Mathematics 27, ISBN 9780387901251, 1975, Springer Verlag.
	
	\bibitem{jkelley2010zh}
John L. 凯莱,  \emph{一般拓扑学}, 数学名著译丛, 吴从炘 / 吴让泉译, ISBN 9787030271181, 2010, 科学出版社.

	\bibitem{PuJiangHu1985}
蒲保明, 蒋继光, 胡淑礼, \emph{拓扑学}, 1985, 高等教育出版社.

	\bibitem{jjiang1991}
蒋继光, \emph{一般拓扑学专题选讲}, ISBN 7540812842, 1991, 四川教育出版社.
	
	\bibitem{enderton1977}
H. B. Enderton, \emph{Elements of Set Theory}, ISBN 9780122384400, 1977, Academic Press.
	
	\bibitem{wang1949}
Hao Wang, \emph{On Zermelo's and von Neumann's axioms for set theory}, Proc. Natl. Acad. Sci. U.S.A., 35 (3): 150–155.
	
\end{thebibliography}
\end{document}
