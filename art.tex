\documentclass[zihao=-4,a4paper]{ctexart}
\usepackage{geometry}
\geometry{a4paper,total={171.8mm,246.2mm},left=19.1mm,top=25.4mm,}
\usepackage{mathrsfs}

\title{凯莱《一般拓扑学》公理集合论附录中\\ 引起悖论的误译}
\author{龙\quad 冰}
\date{}

\begin{document}
\maketitle
\begin{abstract}
科学出版社数学名著译丛 对 John L. Kelley 《 一般拓扑学》一书的 1982 年中文版, 全新排版出版了 2010 年第二版.
本文讨论新版公理集合论附录中引起悖论的误译, 也对另几处商榷.
\end{abstract}
	
关键词: 集合 (set); 类 (class); 序类 (ordinal); 
	序数 (ordinal number); 
	分类公理模式 (classification axiom-scheme).
	
	
	
\section{导言}
科学出版社1982 年在数学名著译丛中首度出版了吴从炘和吴让泉翻译的凯莱  (John L. Kelley) 《 General  Topology》\cite{jkelley1975}一书的中文版 《 一般拓扑学》.时隔 28 年, 2010 年又全新排版出版了中文 第二版\cite{jkelley2010zh}, 可见原著对一般拓扑学领域的发展影响甚大. 反映在教材编写上也是如此\cite{PuJiangHu1985}.译者两版中文译本的工作无疑是值得肯定的. 不太为人知的是, Kelley 该书关于公理集合论附录的重要意义. 该书第 0 章预备知识介绍了一般拓扑学所需的 (非公理化的) 集合论知识, 大部分读者满足于此. 
附录将第 0 章的集合论浓缩式地公理化, 虽然称之为初等集合论, 却艰深难懂. 领略一下, 公理集合论附录的英文原文 31 页. 讲了集合的各种运算和大部分公理, 花了整整 21 页后读者才在无穷公理中见识到了真正的集合存在. 之前只是说, 属于某个类的类是集合, 但没有见到集合, 只是假定, 如果集合存在的话怎样怎样. 比如, 如果集合存在的话, 空类 0 是一个集合,即空集. 如此, 必然有大部分读者敬而畏之, 望而止步. 

事实上, 凯莱《 一般拓扑学》 的公理集合论附录独树一帜, 影响不可忽视. 对 ZFC 公理集合论\cite{jjiang1991}\cite{enderton1977}的扩展主要有三大学派\cite{wikiSetTheory}, 它们都在类下面讨论集合, 而 ZFC 则不直接涉及类. Von Neumann–Bernays–Gödel 集合论\cite{wikiNBGSetTheory}是 ZFC 的保守扩展,  它使用有限公理模式, 得到和 ZFC 相同的结果. 
Morse-Kelley 集合论\cite{wikiMKSetTheory} 和 Tarski–Grothendieck 集合论\cite{wikiTGSetTheory} 都比 ZFC 更强, 后者加了 Tarski 公理以支持范畴论. 凯莱一书的附录概括总结了 Morse-Kelley 集合论,
这方面的研究开始于王浩的工作\cite{wang1949}. Morse-Kelley 集合论与 Von Neumann–Bernays–Gödel  集合论不同, 不能有限公理化.但在它的框架下可以证明 ZFC 和 Neumann–Bernays–Gödel 是一致的.

鉴于凯莱一书公理集合论附录的不应忽视的重要性和本身的艰深难读, 读者自然期望它的中文翻译能妙笔生花, 使名著的阅读更为容易. 对照了一下两版中文, 似乎没有改动. 本文试图纠正已是明显悖论的误译, 也商榷在几处是否可能消除潜在的误解以增加可读性. 虽然有批评, 本文作者对于译者对面艰深名著的翻译所作的专业努力仍表示赞赏.
	
\section{序类和序数 }
	
\section{分类公理模式 }	
有两点需要商榷.
\subsection{Classification axiom-scheme 的翻译}
"Classification axiom-scheme" 被翻译成"分类公理图式". Scheme  是指一种方案、一种计划、一种体系或一种模式, 按照它可以行事. 这里规定以一种特定方式无限种可能地行事的公理, 是一个无限化的公理. 
综述, scheme 译为模式, classification axiom-scheme 称为分类公理模式更为妥当.

\subsection{分类公理模式的表述}
第二条公理的原文是这样表述的:

II. Classification axiom-scheme. An axiom results if ... 意思是说, "如果 ...,
那么一条公理就由此而产生了." 所以, 简洁的翻译是: "此为公理, 如果 ..."
	
译本是这样翻译的: "一个公理的许多结论, 如果..." 没有谓语, 动词改变意思成了复数名词.

\section{其它}

第 176 页最后一行: 故得知  $\bigcap \mathscr{U}$ 没有元, 显然(即定理 26) $\bigcup \mathscr{U}\subset\mathscr{U}$.

\subsection{定理 94 证明的误译(第 182 页 )}

\subsubsection{原著英文} 
{\bf 94 THEOREM} 
{\sl
	If $x$ is an $r$-section of $y$ and $f$ is an $r$-$r$ order-preserving function on $x$ to $y$, 
	then for each $u$ in $x$ it is false that $fu)\, r\, u$.
}

{\bf PROOF}
It must be shown that $\{u: u \in x $ and $ f(u) \, r\, u\}$ is void. 
\underline{If not there is} an $r$-first member $v$ of this class.
Then $f(v)\, r\, v $, and if $u\, r\, v$, then $u\, r\, f(u)$ or $u = f(u)$.
Since $f(v)\, r\, v$, then 
\underline{$f(v)\, r\, f(f(v))$}
or $f(v) = f(f(v))$, but since $f$ is $r$-$r$ order preserving
$f(f(v))\, r\, f(v)$ and this is a contradiction.

\subsubsection{书中误译} 
{\bf 定理 94}\,
{\kaishu
如果 $x$为$y$的一个$r$- 截片且$f$是一个在$x$上到$y$的$r$-$r$保序函
数,则对于在$x$中的每个$u$, $f(u)ru$不真.
}

{\bf 证明}
为了得此定理必须证明$\{u: u \in x $ 且 $ f(u) ru\}$ 是空的. 
\underline{如果不存在}
一个此类的$r$- 首元 $v$, 
则 $f(v) r  v $,并且如果$u r v$,则 $u r f(u)$
或者$u = f(u)$.
由于$f(v)rv$, 于是
\underline{$f(v)r(f(v))$}
或者$f(v)=f(f(v))$. 但是由于$f$ 是$r$-$r$保序的,所以
$f(f(v))rf(v)$, 从而推得一矛盾.

\subsubsection{更正译文} 
{\bf 定理 94}\,
{\kaishu
	如果 $x$为$y$的一个$r$ 截片且$f$是一个在$x$上到$y$的$r$-$r$保序函
	数,则对于在$x$中的每个$u$, $f(u)\, r\, u$不真.
}

{\bf 证明}
必须证明$\{u: u \in x $ 且 $ f(u) \, r\, u\}$ 是空的. 
\underline{如果不空, 那末存在}
该类的一个$r$首元 $v$.
然后 $f(v)\, r\, v $,并且,如果$u\, r\, v$,则 $u\, r\, f(u)$
或者$u = f(u)$.
由于$f(v)\, r\, v$, 于是
\underline{$f(v)\, r\, f(f(v))$}
或者$f(v)=f(f(v))$. 但是由于$f$ 是$r$-$r$保序的,所以
$f(f(v))\, r\, f(v)$, 从而推得一矛盾.

\begin{thebibliography}{9}
	\bibitem{jkelley1975}
John L. Kelley,  \emph{General Topology}, Graduate Texts in Mathematics 27, ISBN 9780387901251, 1975, Springer Verlag.
	
	\bibitem{jkelley2010zh}
J. L. 凯莱,  \emph{一般拓扑学}, 数学名著译丛, 吴从炘 / 吴让泉译, ISBN 9787030271181, 2010, 科学出版社.

	\bibitem{PuJiangHu1985}
蒲保明, 蒋继光, 胡淑礼, \emph{拓扑学}, 1985, 高等教育出版社.

	\bibitem{jjiang1991}
蒋继光, \emph{一般拓扑学专题选讲}, ISBN 7540812842, 1991, 四川教育出版社.
	
	\bibitem{enderton1977}
H. B. Enderton, \emph{Elements of Set Theory}, ISBN 9780122384400, 1977, Academic Press.
	
	\bibitem{wang1949}
Hao Wang, \emph{On Zermelo's and von Neumann's axioms for set theory}, Proc. Natl. Acad. Sci. U.S.A., 35 (3): 150–155.

	\bibitem{wikiSetTheory}
基维百科, 集合论,	https://en.wikipedia.org/wiki/Set\_theory

	\bibitem{wikiMKSetTheory}
基维百科, Morse–Kelley 集合论,	https://en.wikipedia.org/wiki/Morse–Kelley\_set\_theory

   \bibitem{wikiNBGSetTheory}
基维百科, Von Neumann–Bernays–Gödel  集合论,\\ https://en.wikipedia.org/wiki/Von\_Neumann–Bernays–Gödel\_set\_theory

   \bibitem{wikiTGSetTheory}
基维百科, Tarski–Grothendieck  集合论,\\
https://en.wikipedia.org/wiki/Tarski–Grothendieck\_set\_theory

\end{thebibliography}
\end{document}
