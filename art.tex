\documentclass[zihao=-4,a4paper]{ctexart}
\usepackage{geometry}
\geometry{a4paper,total={171.8mm,246.2mm},left=19.1mm,top=25.4mm,}
\usepackage{mathrsfs}
\usepackage[OT2,OT1]{fontenc}

\title{凯莱《一般拓扑学》公理集合论附录中\\ 引起悖论的误译}
\author{龙\quad 冰}
\date{}

\begin{document}
\maketitle
\begin{abstract}
科学出版社数学名著译丛 1982 年 翻译出版了 John L. Kelley 《General Topology》一书的中文版,  2010 年全新排版出版了第二版.
本文讨论新版公理集合论附录中的误译.其中对序类的误译和正则公理抵触而明显产生悖论.
\end{abstract}
	
关键词: 集合 (set); 类 (class); 序类 (ordinal); 
	序数 (ordinal number); 
	分类公理模式 (classification axiom-scheme).
	
	
	
\section{引言}
科学出版社1982 年在数学名著译丛中首度出版了吴从炘和吴让泉翻译的凯莱  (John L. Kelley) 《 General  Topology》\cite{jkelley1975}一书的中文版 《 一般拓扑学》.时隔 28 年, 2010 年又全新排版出版了中文 第二版\cite{jkelley2010zh}, 可见原著对一般拓扑学领域的发展影响甚大. 反映在教材编写上也是如此\cite{PuJiangHu1985}.译者两版中文译本的工作无疑是值得肯定的. 不太为人知的是, 凯莱该书关于公理集合论附录的重要意义. 该书第 0 章预备知识介绍了一般拓扑学所需的 (非公理化的) 集合论知识, 大部分读者满足于此. 
附录将第 0 章的集合论浓缩式地公理化, 虽然称之为初等集合论, 却艰深难懂. 领略一下, 公理集合论附录的英文原文 31 页. 在讲集合的各种运算和大部分公理, 花了整整 21 页后
才在无穷公理中真正确定了有集合存在. 之前只是说, 属于某个类的类是集合. 只是假定, 如果集合存在的话怎样怎样. 比如, 如果集合存在的话, 空类 0 是一个集合,即空集. 如此, 必然有大部分读者敬而畏之, 望而止步. 

事实上, 凯莱《 一般拓扑学》 的公理集合论附录独树一帜, 影响不可忽视. 对 ZFC 公理集合论\cite{jjiang1991}\cite{enderton1977}的扩展主要有三个学派\cite{wikiSetTheory}, 他们都在类下面讨论集合, 而 ZFC 则不直接涉及类. Von Neumann–Bernays–Gödel 集合论\cite{wikiNBGSetTheory}是 ZFC 的保守扩展,  它使用有限公理模式, 得到和 ZFC 相同的结果. 
Morse-Kelley 集合论\cite{wikiMKSetTheory} 和 Tarski–Grothendieck 集合论\cite{wikiTGSetTheory} 都比 ZFC 更强. 后者加了 Tarski 公理以支持范畴论. 凯莱一书的附录则概括总结了 Morse-Kelley 集合论,
这方面的研究开始于王浩的工作\cite{wang1949}. Morse-Kelley 集合论与 Von Neumann–Bernays–Gödel  集合论不同, 不能有限公理化.但在它的框架下可以证明 ZFC 和 Neumann–Bernays–Gödel 是一致的.

鉴于凯莱一书公理集合论附录的不应忽视的重要性和本身的艰深难读, 读者自然期望它的中文翻译能妙笔生花, 使名著的阅读更为容易. 对照了一下两版中文, 似乎没有改动. 本文试图纠正明显的误译, 也商榷在几处是否可能消除潜在的误解以增加可读性. 严重的是,对序类和序数不加区分的误译产生悖论.虽然有批评, 本文作者对于译者面对艰深名著的翻译所作的专业努力仍表示赞赏.

为了方便阅读,本文在原文、误译和译文更正的相关部分加了下划线.
	
\section{不能混淆序类和序数}
\subsection{Ordinal 是序类, Ordinal Number 才是序数}
在Morse-Kelley 公理集合论里,一个集合是满足属于关系$\in$ 的一种特殊的类.见\cite{jkelley2010zh}第173页的

{\bf 定义 1}\, {\kaishu $x$为一集合当且仅当对于某一 $y$, $x\in y$. }

在确定$x\in y$之前, $x$只能叫做类, 不能叫集合.
附录第A8 节原文的标题是 Ordinals,应当是序类,而不应当误译为序数. 该节首先给出正则公理 IV. 附录在定理 38 之后(\cite{jkelley2010zh}第177页) 指出, 自己属于自己这种逻辑自指 (self-reference) 会导致 
罗素(Russell) 悖论. 正则公理的目的是防止集合论中出现 $x\in x$ 或 
$x\in x_1 \in x_2 \in \cdots \in x_n \in x$
之类的逻辑自指. 该节从第一个定义/定理 (定理101)开始一直到 定义 115 之前, 都未自动假设 一个 ordinal $\in$ 另一个类. 所以一个 ordinal 是一个类, 但不一定是一个集合.
事实上, 原著(\cite{jkelley1975} 第268 - 269页)有

{\bf 113 THEOREM}\,
{\sl 
	$R$ is an \underline{ordinal} and $R$ is not a set.
}

\noindent 与 ordinal 不是同, 原著定义 ordinal number 如下

{\bf 115 DEFINITION}\,
{\sl 
	$x$ is an ordinal number if and only $x \in R$.
}

\noindent 重大区别在于, 一个 ordinal number 必定是一个集合, 原因是属于关系 $\in$.

但是, 中译本不加区别, 把 ordinal  和 ordinal number 都译成了序数. 正确的翻译是, 单纯的 ordinal 应当译为{\kaishu 序类}, 它不一定是集合. {\kaishu 序数}一词应当保留给 ordinal number, 它必定是一个集合. 
如果按不加区别的错误译法, 在下面两段(\cite{jkelley2010zh}第185页) 中, 从错译的定理和正确翻译的定义,我们看到的是悖论: 一个序数不是集合而同时又必定是一个集合.

{\bf 定理 113}
{\kaishu
$R$是一个\underline{序数},但不是一个集.
}

{\bf 定义 115}
{\kaishu
$x$ 是一个序数当且仅当$x\in R$.
}

\noindent 显然, 定理 113 需要更正为:

{\bf 定理 113}\, 
{\kaishu
	$R$是一个\underline{序类},但不是一个集合.
}

\noindent
定理 113 的翻译容不得出错,因为凯莱对此专门加了一个注: 
“这个定理实质上是 Burali-Forti悖论的叙述
- 在历史上是直观集合论的第一个悖论."

整个这一节都应当加以更正,才能不抵触正则公理 IV. 这里就不详细列出更正了.

\subsection{定理 140 的证明的误译}
下面的定理作为选择公理的应用而被证明. 它对基数的定义和大小的可比较性是至关重要的. 可是, 证明的翻译有严重错误.

{\bf 定理 140} \,  
{\kaishu
如果$x$是一个集,存在1-1 函数,它的值域是$x$而它的定义域是
一个序数.
}

在\cite{jkelley2010zh}第 188 页的证明中, 

1) 第14 行, “应用定理 128,存在函数$f$使得$f$的定义域是一个\underline{序数}" 
是错误的翻译, 源自对之前的定理 128的错误翻译.对误译(\cite{jkelley2010zh}第186页)更正后有

{\bf 定理 128} \,
{\kaishu
对于每个$g$存在唯一的函数$f$使得$f$的定义域是一个序类,并且
对于每个序数$x, f(x) = g(f\vert x)$.
}

\noindent
根据原文以及对定理 128的正确翻译,
此处应为“应用定理 128,存在函数$f$, 使得$f$的定义域是一个\underline{序类}". 要待验证以下两条以后,我们才能确定, $f$的定义域是一个\underline{序数}.在这一之前只能说,$f$的定义域是一个\underline{序类}.

2) 第14 行,  我们见到了正确地翻译出的理由之一: “所以$f$的定义域
$=R$是不可能的".

3) 第18 行,  “ 从而$f$的定义域 $=R$", 它的明显跟上条2)矛盾, 是翻译错误. 根据原文, 
应译为: “ 从而$f$的定义域 $\in R$".


\section{分类公理模式 }	
译本(\cite{jkelley2010zh}第174页)有两点需要商榷.
\subsection{Classification axiom-scheme 的翻译}
“Classification axiom-scheme" 被翻译成“分类公理图式". Scheme  是指一种方案、一种计划、一种体系或一种模式, 按照它可以行事. 这里规定以一种特定方式无限种可能地行事的公理, 是一个无限化的公理. 
综述, scheme 译为模式, classification axiom-scheme 称为分类公理模式更为妥当.

\subsection{分类公理模式的表述的翻译}
公理的原文(\cite{enderton1977}第253页)是这样表述的:

{\bf II. Classification axiom-scheme.} An axiom results if ... 

\noindent 意思是说, “如果 ...,
那么一条公理就由此而产生了." 所以, 简洁的翻译是: “此为公理, 如果 ..."
这也跟俄文版(\cite{jkelley1968ru} 第 329页)是一致的: 

{\bf II}  {\fontencoding{OT2}\selectfont 
Klassifika{ts}ionna{ya}   {s}hema aksiom.
	My poluqaem aksiomu, esli v ...
}

\noindent 俄文转译成中文是: “我们得到一条公理, 如果 ..."
所以, 译本这样的翻译不恰当的,值得商榷: “一个公理的许多结论, 如果..." 没有谓语, 动词改变意思成了复数名词.







\section{定理 94 证明的误译}

\subsection{原著英文(\cite{jkelley1975}第264页)} 
{\bf 94 THEOREM} 
{\sl
	If $x$ is an $r$-section of $y$ and $f$ is an $r$-$r$ order-preserving function on $x$ to $y$, 
	then for each $u$ in $x$ it is false that $fu)\, r\, u$.
}

{\bf PROOF}
It must be shown that $\{u: u \in x $ and $ f(u) \, r\, u\}$ is void. 
\underline{If not there is} an $r$-first member $v$ of this class.
Then $f(v)\, r\, v $, and if $u\, r\, v$, then $u\, r\, f(u)$ or $u = f(u)$.
Since $f(v)\, r\, v$, then 
\underline{$f(v)\, r\, f(f(v))$}
or $f(v) = f(f(v))$, but since $f$ is $r$-$r$ order preserving
$f(f(v))\, r\, f(v)$ and this is a contradiction.

\subsection{书中误译(\cite{jkelley2010zh}第 182 页 )} 
{\bf 定理 94}\,
{\kaishu
如果 $x$为$y$的一个$r$- 截片且$f$是一个在$x$上到$y$的$r$-$r$保序函
数,则对于在$x$中的每个$u$, $f(u)ru$不真.
}

{\bf 证明}
为了得此定理必须证明$\{u: u \in x $ 且 $ f(u) ru\}$ 是空的. 
\underline{如果不存在}\! 一个此类的$r$- 首元 $v$, 
则 $f(v) r  v $,并且如果$u r v$,则 $u r f(u)$
或者$u = f(u)$.
由于$f(v)rv$, 于是
\underline{$f(v)r(f(v))$}
或者$f(v)=f(f(v))$. 但是由于$f$ 是$r$-$r$保序的,所以
$f(f(v))rf(v)$, 从而推得一矛盾.

\subsection{更正译文} 
{\bf 定理 94}\,
{\kaishu
	如果 $x$为$y$的一个$r$ 截片且$f$是一个在$x$上到$y$的$r$-$r$保序函
	数,则对于在$x$中的每个$u$, $f(u)\, r\, u$不真.
}

{\bf 证明}
必须证明$\{u: u \in x $ 且 $ f(u) \, r\, u\}$ 是空的. 
\underline{如果不空, 那末存在}\! 该类的一个$r$首元 $v$.
然后 $f(v)\, r\, v $,并且,如果$u\, r\, v$,则 $u\, r\, f(u)$
或者$u = f(u)$.
由于$f(v)\, r\, v$, 于是
\underline{$f(v)\, r\, f(f(v))$}
或者$f(v)=f(f(v))$. 但是由于$f$ 是$r$-$r$保序的,所以
$f(f(v))\, r\, f(v)$, 从而推得一矛盾.


\section{其它勘误}

\begin{tabular}{l l l l }
	\hline
{\kaishu 译文页/行} & {\kaishu 原著英文} & {\kaishu 误译} & {\kaishu 更正译文} \\
	\hline
182 / -2 & $g^{-1}$ is order preserving  & $f^{-1}$ 是保序的 & $g^{-1}$ 是保序的 \\
184 / 3 & The class $E$ is the $\in$-relation. &类$E$是 $E$关系. & 类$E$是 $\in$关系. \\
多处 & $E$-first member & $E\_ $ 首元 &  $E$ 首元 \\
多处 & $E$-last member & $E\_ $ 末元 &  $E$ 末元 \\
	\hline
\end{tabular}



%第 176 页最后一行: 故得知  $\bigcap \mathscr{U}$ 没有元, 显然(即定理 26) $\bigcup \mathscr{U}\subset\mathscr{U}$.



\begin{thebibliography}{9}

	\bibitem{jkelley1975}
John L. Kelley,  \emph{General Topology}, Graduate Texts in Mathematics 27, ISBN 9780387901251, reprint 1975, Springer Verlag.
	
	\bibitem{jkelley2010zh}
J. L. 凯莱,  \emph{一般拓扑学}, 数学名著译丛, 吴从炘 / 吴让泉译, ISBN 9787030271181, 2010, 科学出版社.

	\bibitem{jkelley1968ru}
{\fontencoding{OT2}\selectfont  D{zh}. L. Kellei, Ob{shch}a{ya} topologi{ya}, Nauka,} 1968.
%Келли Дж. Л.[en] Общая топология — М.: Наука, 1968


	\bibitem{PuJiangHu1985}
蒲保明, 蒋继光, 胡淑礼, \emph{拓扑学}, 1985, 高等教育出版社.

	\bibitem{jjiang1991}
蒋继光, \emph{一般拓扑学专题选讲}, ISBN 7540812842, 1991, 四川教育出版社.
	
	\bibitem{enderton1977}
H. B. Enderton, \emph{Elements of Set Theory}, ISBN 9780122384400, 1977, Academic Press.
	
	\bibitem{wang1949}
Hao Wang, \emph{On Zermelo's and von Neumann's axioms for set theory}, Proc. Natl. Acad. Sci. U.S.A., 35 (3): 150–155.

	\bibitem{wikiSetTheory}
基维百科, 集合论,	https://en.wikipedia.org/wiki/Set\_theory

	\bibitem{wikiMKSetTheory}
基维百科, Morse–Kelley 集合论,	https://en.wikipedia.org/wiki/Morse–Kelley\_set\_theory

   \bibitem{wikiNBGSetTheory}
基维百科, Von Neumann–Bernays–Gödel  集合论,\\ https://en.wikipedia.org/wiki/Von\_Neumann–Bernays–Gödel\_set\_theory

   \bibitem{wikiTGSetTheory}
基维百科, Tarski–Grothendieck  集合论,\\
https://en.wikipedia.org/wiki/Tarski–Grothendieck\_set\_theory

\end{thebibliography}
\end{document}
